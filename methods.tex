For the full proposal, we will perform two studies.  One will target small scale
structures using simulations of driven magnetized turbulence.  The other will
target large scale features and variation by simulating isolated galaxies.  Our
scaling study will reflect these two use cases.  


The code we will use is Enzo \citep{Collins10,Bryan14}, an open source code that
has been used for a number of astrophysics applications
\citep{Abel02,CorreaMagnus23}.  Enzo is an adaptive mesh refinement (AMR) code
that dynamically adds resolution elements as the system requires it.  We will
use the constrained transport (CT) module \citep{Collins10} that conserves the
divergence of the field to machine precision.  It uses FFT-based gravity for the
root grid and multigrid relaxation for gravity on fine grids.  The base MHD
solver is a higher order Godunov method.  


There are three major numerical systems that need to be profiled:  the hydro
solver, the gravity solver, and the AMR machinery.  The hydro solver alone is
entirely local, and has been demonstrated to scale quite well.  The gravity
solver uses an additional FFT to solve the gravity which will have its own
scaling properties.  Finally the AMR machinery that communicates the solution
between coarse and fine patches scales less well than the other two.  We will
thus perform three suites of weak scaling with the three dominant packages.  The
MHD suite will use only the MHD solver; the Gravity suite will use MHD and
gravity; and finally the AMR suite will incorporate AMR as well as MHD and
gravity.  In practice the AMR grid layout is chaotic.  For simplicity and
reproducibility we will use three levels, with the two refined regions covering
half the box length (1/8 of the total volume), keeping the same number of zones
per level.  None of the main numerical packages have costs that depend on the
state of the gas, so we will use uniform density and temperature.

The cost of the scaling study, $SU$, is found as 
\begin{align}
SU &= t_{\rm{wall}} N_{N}\\
t_{\rm{wall}} &= \frac{N_{Z} N_{U}}{\zeta} \frac{1}{N_{C}},
\end{align}
where $\zeta$ is the performance of the code in zone-updates/core-second, $N_Z$
is the number of zones, $N_U$ is the number of updates, $N_{C}$ is the number of
cores, and $N_N$ is the number of nodes.  The performance, $\zeta$, is
in principle independent of the size of the problem, and only depends on the
numerical package used.  This is unknown on Frontera, and its measurement in fact the purpose
of the scaling study.  We will use the values obtained from previous scaling
studies on \emph{Stampede 2}, which will be somewhat lower than Frontera but
will give the right ballpark.  

We will use 56 cores per node and fixed work of $64^3$ root grid zones per core.
The number of root grid zones will be $256^3$, $512^3$, $1024^3$ and $2048^3$.
This leaves one node underutilized for each run, which will not impact our
results much.  We will take 100 root grid time steps for each of our
simulations.  For the AMR suite, each subgrid takes on average two steps per
parent.  Since all three levels have the same number of zones,  this is
equivalent to 700 steps of one level.  
This number of time steps was chosen to account for the inevitable repetition
of the suite as we optimize our build for Frontera.

\begin{table} \begin{center} \caption{The request broken down by physics suite (first colums).  $N_Z$ is the
number of zones.  The number of cores per node will be 56, and $64^3$ zones per
grid will be used.  $\zeta$ is the performance of the code in zone-updates per
core-seconde, here estimated on \emph{Stampede 2}. $N_U=100$ is the number of
root grid updates.}
 \label{table2}                                                                                                               
\begin{tabular}{               l               r               r               r               r                              }
                   suite       &$N_Z$       &   Nodes       &   $\zeta$       &      SU      \\                       
                  \hline       &               &               &               &              \\                       
                MHD       &   $ 256^3$     &       2       &1.0\sci{5}       &    0.15      \\                       
                MHD       &   $ 512^3$     &      10       &1.0\sci{5}       &    0.73      \\                       
                MHD       &   $1024^3$     &      74       &1.0\sci{5}       &    5.39      \\                       
                MHD       &   $1536^3$     &     247       &1.0\sci{5}       &   17.99      \\                       
                MHD       &   $2048^3$     &     586       &1.0\sci{5}       &   42.67      \\                       
                  \hline       &               &               &               &              \\                       
                 Gravity       &   $ 256^3$      &       2       &1.3\sci{4}       &    1.10      \\                       
                 Gravity       &   $ 512^3$      &      10       &1.3\sci{4}       &    5.52      \\                       
                 Gravity       &   $1024^3$      &      74       &1.3\sci{4}       &   40.82      \\                       
                 Gravity       &   $1536^3$      &     247       &1.3\sci{4}       &  136.26      \\                       
                 Gravity       &   $2048^3$      &     586       &1.3\sci{4}       &  323.27      \\                       
                  \hline       &               &               &               &              \\                       
                     AMR       &   $ 256$      &       2       &1.2\sci{4}       &    8.50      \\                       
                     AMR       &   $ 512$      &      10       &1.2\sci{4}       &   42.48      \\                       
                     AMR       &   $1024$      &      74       &1.2\sci{4}       &  314.33      \\                       
                     AMR       &   $1536$      &     247       &1.2\sci{4}       & 1049.18      \\                       
                     AMR       &   $2048$      &     586       &1.2\sci{4}       & 2489.15      \\                       
                  \hline                                                                      \\                       
                               &               &               &   Total       & 4477.53      \\                       
                                                                                                                       
\end{tabular}                                                                                                               
\end{center}                                                                                                               
\end{table}                                                                                                                


Table \ref{table2} shows a breakdown of the cost of the scaling study, and
includes the value of $\zeta$ used for each suite.  The first column shows the
package, the MHD solver, MHD with Gravity, and AMR with MHD and Gravity.  
The second column shows the number
of zones per run, the third column shows the number of nodes to be used, the
fourth column shows the anticipated performance, $\zeta$, and the final column
shows the total cost.  


